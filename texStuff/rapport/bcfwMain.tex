\subsection{Block Coordinate Frank Wolfe}
Due to the exponential number of dual variables in the structured SVM setting,
classical algorithms like projected gradient are intractable. Stochastic
subgradient methods, on the other hand, achieve a sublinear convergence rate
while only requiring a single call to the maximization oracle every step. They
are nonetheless very sensitive to the sequence of stepsizes and it is unclear
when to terminate the iterations. \\

In fact many algorithms with good theoretical garantees introduce a dependency
between the stepsize to be chosen and some constants which caracterize the
function, e.g. the lipschitz constant of the gradient ($L$) and the
strong-convexity constant ($\mu$), they are however not necessarily computable.
One of the main interests in FW approaches is that we get an optimality
certficate \textit{for free} which ties in elegantly with convex analysis
results. Moreover we have many good options for the choice of step-size with
theoretical guarantees.

Leveraging those aspects the Block-Separable Frank Wolfe algorithm proposed by
\cite{lacoste-julienBlockCoordinateFrankWolfeOptimization2013} has a stepsize
which can be computed exactly analytically and a computable duality gap while
still retaining a sublinear convergence rate. Moreover, despite the exponential
number of constraints, the algorithm has sparse iterates alleviating the memory
issues which come with the exponential number of dual variables.

We are considering seperability of an algorithm hence the following well known
result is essential:
\begin{theorem} Given a convex, differentiable objective
$f:\mathcal{M}^{1}\times...\times\mathcal{M}^{n}\to\mathbb{R}$, where $\forall
i\in\{1..n\}$, each factor\quad $\mathcal{M}^{i}\subseteq\mathbb{R}^{n}$ is
convex and compact, if we are at a point $x$ such that $f(x)$ is minimized along
each coordinate axis, then $x$ is a global minimum.
\end{theorem}


The BCFW algorithm as presented in
\citet{lacoste-julienBlockCoordinateFrankWolfeOptimization2013}

\begin{algorithm}
  \caption{Block Seperable Frank-Wolf}
  \label{alg:bcgfw}
\begin{algorithmic}
   \STATE Let $\alpha\in\mathcal{M}$
    \STATE {Let $w^{0}= 0$, $l^{0}= 0$}\\
    \FOR{$k=0, \dots, K$}
      \FOR{$i=1, \dots, n$}
        \STATE {Solve $y_{i}^{*}=\underset{y_{i}\in\mathcal{Y}_{i}}{\max} H_{i}(y,w^{k})$}\\
        \STATE {Let $w_{s}= \sum_{i=1}^{n}\frac{1}{n\lambda}\psi_{i}(y_{i}^{*})$,
                and $l_{s}= \frac{1}{n}\sum_{i=1}^{n}L_{i}(y_{i}^{*})$}\\
        \STATE {Let $\gamma= \frac{\lambda(w^{k}-w_{s})^{T}w^{k}- l^{k}+ l_{s}}{\lambda||w^{k}-w_{s}||^{2}}$,
                and clip to $[0,1]$}\\
        \STATE {Update $w^{k+1}= (1-\gamma)w^{k}+ \gamma w_{s}$,
                and $l^{k+1}= (1-\gamma)l^{k}+ \gamma l_{s}$}\\
      \ENDFOR
   \ENDFOR
\end{algorithmic}
\end{algorithm}


% \begin{proof}
% The objective function being differentiable and convex, if we are at a point
% $\alpha$ such that $f(\alpha)$ is minimized along each coordinate axis, then
% $\alpha$ is a global minimizer. Therefore,
% \begin{align}
%     &\underset{s\in\mathcal{M}}{\textit{min}}\langle s, \nabla f(\alpha)\rangle
% = \sum_{i}\underset{s_{i}\in\Delta_{|\mathcal{Y}_{i}|}}{\textit{min}}\langle
% s_{i}, \nabla_{i} f(\alpha)\rangle
% \end{align}

% Moreover, with
% \begin{align*}
%    &w=A\alpha, A=\Big[\frac{1}{n\lambda}\psi_{1}(y)...\frac{1}{n\lambda}\psi_{\sum_{i}|\mathcal{Y}_{i}|}(y)\Big]\\
%    &\textit{and}\quad b=\Big(\frac{1}{n}L_{i}(y)\Big)_{i\in\big[n\big],y\in\mathcal{Y}_{i}}
% \end{align*}
% The gradient of the dual would be,
% \begin{align*}
%     &\nabla f(\alpha)= \nabla\Big[\frac{\lambda}{2}||A\alpha||^{2}- b^{T}\alpha\Big] = \lambda A^{T}A\alpha- b\\
%     &= \lambda A^{T}w- b= \frac{1}{n}H_{i}(y,w)\\
%     &\underset{y_{i}\in\mathcal{Y}_{i}}{\textit{max}}\quad\tilde{H}_{i}=
%   -\underset{y_{i}\in\mathcal{Y}_{i}}{\textit{min}}\quad\tilde{H}_{i} =
%   \underset{y_{i}\in\mathcal{Y}_{i}}{\textit{min}}\quad L_{i}- \langle w,
%   \psi_{i}\rangle\\ &=
%   \underset{s_{i}\in\Delta_{|\mathcal{Y}_{i}|}}{\textit{min}}\langle s_{i},
%   \nabla_{i} f(\alpha)\rangle\\
% \end{align*}
% Thus we can see that, if $n=$ size of the training data, one Frank-Wolfe step is
% equivalent to $n$ calls to the maximization oracle.
% \end{proof}

%%% Local Variables:
%%% mode: latex
%%% TeX-master: "mainProject"
%%% End:

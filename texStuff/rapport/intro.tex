Structured prediction in machine learning is tasked with learning
predictors where the labels on the datapoints are more than simple
tags but have inherent structure which implies the following:
\begin{itemize}
\item The number of potential labels for a given feature vector
  can grow exponentially with the input which makes traditional
  classification procedures intractable
\item A certain intelligibility of the structure; hence a hope to
  leverage it to improve tractability
\end{itemize}
It is often quite hard to know in advance wether we can tackle a structured
prediction problem with known approaches.

In this paper we discuss two different approaches to solving structured
prediction problems, both focusing on a \emph{large-margin approach} which
translates into a support-vector machine - like problem formulation.

Let us introduce some notation which we will use throughout the paper.
Define the dataset to be
\begin{equation}
  S = \{ \xiii, \yiii) \}_{i=1}^{n} \in (\mathcal X \times \mathcal Y)^{n} \label{intro:eq:dataset}
\end{equation}

If one were to frame the machine learning goal in the concisest and simplest way
possible we could say that the goal is to \emph{learn} a (parametrized) \emph{predictor}
function:
\begin{align*}
  h_w \quad & : \quad  \mathcal{X} \rightarrow \mathcal Y\\
  &: \quad \hat x \mapsto y \qquad y\in \mathcal{Y} (\hat x)
\end{align*}
where $\hat x$ is just some arbitrary sampled $\hat x$ and we abuse notation and mean
the set valued mapping which outputs the feasible label set for a given $x$ by
$\mathcal Y(x)$

The traditional probabilistic approach is to compute $h$ or its parameter by
calculating the most likely parameter given conditionned on the observed data.
However this is more often than not untractable in the structured prediction
context as summed up in \citet{taskarStructuredPredictionDual2006}

In a large margin approach we wish to compute the predictor
function the following way :
\begin{equation}
  h_{w}(x)= \arg\max_{y\in \mathcal Y} \, \langle w, \phi(x,y)\rangle
\end{equation}
where $\phi$ is just the feature map for the dataset.

This is a constrained optimization problem and the structure of $\mathcal Y$
clearly has a big influence on how well we can solve the problem as
well as which method should work well. 

The dual extragradient approach presented in
\citet{taskarStructuredPredictionDual2006} leverages
a saddle-point approach to the problem to tackle the problem
with a first order method and moreover ``this approach
allows us to exploit the structure of W and Z separately, allowing for efficient solutions for a wider
range of parameterizations and structures.''\citep{taskarStructuredPredictionDual2006}

\clearpage
Despite its improvements at the time the dual extragradient approach suffers
from two main computational draw-backs:
\begin{enumerate}
\item It is not seperable/'stochasizable' which is a severe problem as
  machine learning has seen much of its success in recent years by optimizing
  through first-order sampled method, i.e. first order method which
  use a probabilistic approximation of the gradient through sampling instead
  of computing the actual gradient (which might be unfeasible).
  TODO : cite mr simon

\item  The algorithm requires a projection step which can be expensive depending
  on the structure of $\mathcal Y$.
  TODO : cite explain why proj is expensive
\end{enumerate} 
 
This leads us to consider the Block Seperable Frank Wolfe algorithm
presented in \citet{lacoste-julienBlockCoordinateFrankWolfeOptimization2013}




%%% Local Variables:
%%% mode: latex
%%% TeX-master: "mainProject.tex"
%%% End:

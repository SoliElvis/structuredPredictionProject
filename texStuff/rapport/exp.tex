\subsection{Experiments}
In this section, we describe the implementation we performed for this project.
The goal was to see how the Dual Extragradient algrithm compared to the
Block-coordinate Frank-Wolfe algorithm. The task on which we performed the
evaluation was word alignment in machine translation. We extracted the dataset
from the Europarl dataset \textbf{CITE THIS SHIT}. The data consisted of
approximately 2 million sentence pairs in both english and french. Each the
sentences in each pair were translations of one another. The data had been
aigned using the GIZA++ algorithm \textbf{CITE THIS SHIT}. We extracted all
sentences and performed a clean by splitting longer sentences into shorter ones.
The goal of this step was to reduce the eventual number of matchings in
training, which could take long to solve using a LP solver.

We then proceeded to implement the Dual Extragradient and BCFW algorithms using
a SVM. We had to define a feature mapping for an input sentence pair. As in
Taskar \cite{taskarStructuredPredictionExtragradient}, the features were composed of the following
statistics of the corpora:
\begin{itemize}
  \item Dice coefficient between pairs of words:
    \begin{equation*}
      \frac{2*C_{ef}}{C_e + C_f}
    \end{equation*}
    where $C_ef$ is the number of co-occurrences of the pair of words $e,f$ in
sentences and $C_e, C_f$ are the number of occurrences of the words in english
and french respectively.

  \item Absolute value of the difference of the relative positions of each words in each sentence

  \item The square root of that value

  \item The square of the same value

  \item The hamming distance between the two words in a pair (padded with zeros when the lenght is not the same)

\end{itemize}
We combined these statistics by summing them for all word tuples in each sentence pairs. As an example, consider the following two sentences:
\begin{itemize}
  \item This assignment was hard
  \item Ce travail etait ardu
\end{itemize}
We would compute the statistics for each word tuple (e.g. this/ce, this/travail,
\dots, hard/ardu). Thus, we end up with a vector of statistics for each tuple.
Then, we simply performed a weighted combination of these vectors using the edge
labels as weights. To clarify what we mean by edge labels, suppose that the edge
linking ``assignment'' and ``travail'' has a value of 1. Then, we weight the
vector extracted from these two words with 1. As another example, if the edge
between ``ce'' and ``ardu'' has a value of 0 (i.e. no link between the words),
we do not include the vector computed from the statistics of this word pair.
This is exactly what was done in Taskar \textbf{CITE THIS SHIT} modulo some
other features.

In the implementation of BCFW, we used the solver from scipy \textbf{CITE THIS
SHIT} with the simplex method. Since the constraint matrix given by the
optimization of $H_i$ in the algorithm is unimodular, when we relax the LP, we
still get a solution to the ILP without relaxation. Thus, we take advantage of
this fact and indeed use the LP solver. The loss that we used was the $L_1$
distance between the two labels, the proposal and the ground truth.

\subsubsection{Results}

%%% Local Variables:
%%% mode: latex
%%% TeX-master: t
%%% End:

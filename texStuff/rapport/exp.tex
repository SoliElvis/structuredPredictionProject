\subsection{Experiments}

In this section, we describe the implementation we performed for this project.
The goal was to see how the Dual Extragradient algrithm compared to the
Block-coordinate Frank-Wolfe algorithm. The task on which we performed the
evaluation was word alignment in machine translation. We extracted the dataset
from the Europarl dataset \cite{EuroparlParallelCorpus}. The data consisted of
approximately 2 million sentence pairs in both english and french. Each the
sentences in each pair were translations of one another. We extracted all
sentences and performed a clean by splitting longer sentences into shorter ones.
The goal of this step was to reduce the eventual number of matchings in
training, which could take long to solve using a LP solver. Of course, each
sentence was tokenized beforehand.

We then proceeded to implement the Dual Extragradient and BCFW algorithms using
a SVM. We had to define a feature mapping for an input sentence pair. The features
were extracted using the fastText library \cite{fastText}. This library included
a model that was previously trained to learn embeddings of words in both english
and french. We later combined the embeddings in the two languages by applying a
transformation found in \citet{chojnackiRandomGraphGenerator2010}. This transformation consisted in applying
a matrix to each vector in each language (matrix $W$ for english and $Q$ for french). 
These matrices are in fact orthogonal (i.e. $Q^T W = I$). The idea behind such a transformation
is that we sort of ``put" or ``align" both languages in the same vector space, a sort of ``middle ground". 
This way we can better compare the words ``cat" and ``chat" by getting their cosine similarity measure. 
We combined the cosine measure of each pair of words in the alignment by summing. The following blog post
\citet{AligningVectorRepresentations2017} provides a good intuition using maps that are aligned.
As an example, consider the following two sentences:
\begin{itemize}
  \item This assignment was hard
  \item Ce travail etait ardu
\end{itemize}
We would compute the cosine distance for each word tuple (e.g. this/ce, this/travail,
\dots, hard/ardu). We were then able to get the highest match of each english word for
for french translation. This was how we extracted the ``labels". As the dataset was
not annotated with alignments, we had to compute those according to the procedure mentioned.

For the features, we used the concatenated
embeddings of each word pairs in the alignment. This gave us our edge score.
Then, we simply performed a weighted combination of these vectors using the edge
labels as weights. To clarify what we mean by edge labels, suppose that the edge
linking ``assignment'' and ``travail'' has a value of 1. Then, we weight the
vector extracted from these two words with 1. As another example, if the edge
between ``ce'' and ``ardu'' has a value of 0 (i.e. no link between the words),
we do not include the vector computed from the statistics of this word pair.
This is exactly what was done in Taskar
\cite{taskarStructuredPredictionExtragradient} modulo some other features.

In the implementation of BCFW, we used the solver from scipy with the simplex
method. Since the constraint matrix given by the optimization of $H_i$ in the
algorithm is unimodular, when we relax the LP, we still get a solution to the
ILP without relaxation. Thus, we take advantage of this fact and indeed use the
LP solver. The loss that we used was the $L_1$
distance between the two labels, the proposal and the ground truth.

\subsubsection{Results}
Since running the experiments was computationally intensive and we did not dispose of a
lot of computing power, we had to restrict the training set to a number a relatively small
number of sentence pairs (100). This sanity check was simply a hint to the general applicability 
of the method as we scale up the number of training examples. We used the default $\lambda$ value
of $0.01$ as our regularizer since, we did not want to train for too many iterations before
getting decent results as per Theorem 3 found in
\citet{lacoste-julienBlockCoordinateFrankWolfeOptimization2013}. We were able to
get the following results for the BCFW algorithm:



For the dual extragradient, we obtained instead the following:


%%% Local Variables:
%%% mode: latex
%%% TeX-master: "mainProject"
%%% End:

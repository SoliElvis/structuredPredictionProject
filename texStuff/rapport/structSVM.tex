\subsection{Structured SVM N-slack formulation}
Let us recall the basic goal; to construct an accurate linear
classifier\footnote{clearly we can easily extend to non-linear cases, with kernel maps for example}
\begin{equation}
  h_{w}(x)= \arg\max_{y\in \mathcal Y} \, \langle w, \phi(x,y)\rangle
\end{equation}

In general finding the optimal seperating hyperplane is an ill-defined problem.
In the support vector machine (SVM) setting we want to find the seperating
hyperplane with the \emph{largest margin}. The support vectors are the
datapoints which lie on the the defined margins. It is not obvious
that we should strive to completely seperate the sets so we can include some
slack variables. Indeed, some set of points might not be linearly separable but
adding slack variables might allow one to misclassify some point but get 
a higher proportion of correctly classified points.\\

We call the following way to pose problem the $n$-slack formulation
\footnote{see \citet{moguerzaSupportVectorMachines2006} for example} of the problem :
\begin{align}
    &\max_{\omega, \xi}\quad\frac{\lambda}{2}||w||^{2}+ \frac{1}{n}\sum_{i=1}^{n}\varepsilon_{i}\\
    &\textit{s.t.}\quad \langle w, \psi_{i}(y)\rangle \geq L(y_{i},y)-
\varepsilon_{i},\quad\forall i ,\forall y \in\mathcal{Y}(x)=\mathcal{Y}_{i}
\end{align}

\subsection{Loss-Augmented Decoding}
Define $\psi(y) := \phi(\xiii,\yiii) - \phi(\xiii,y)$ and $L_i(y) = L(\yiii, y)$
\begin{align}
H_i &= \max_{y\in\mathcal{Y}_{i}} \quad \{L_{i}(y)- \langle w, \psi_{i}(y)\rangle\} \\
\mathcal H_i &= \argmax_{y\in\mathcal{Y}_{i}} \quad \{ L_{i}(y)- \langle w, \psi_{i}(y)\rangle \}
\end{align}

\textbf{Proposition.} The max oracle is a convex upper bound to the task loss.\\
\textit{Proof sketch:} The maximum of two convex (linear) functions is convex, and
\begin{equation*}
\begin{align} L(y_{i},h_{w}(x_{i})) &\leq L(y_{i},h_{w}(x_{i})) +
        \underbrace{\langle w, \psi_{i}(y)\rangle}_{\geq 0 \textit{ by definition}} \\
    \quad\quad &\leq \max_{y\in\mathcal{Y}_{i}} L_{i}(y)- \langle w, \psi_{i}(y)\rangle
\end{align}
\end{equation*}

It is not hard to see thus that learning $w$ amounts to the unconstrained problem,
\begin{equation*}
\begin{aligned}
    \max_{w} \quad\frac{\lambda}{2}||w||^{2}+ \frac{1}{n}\sum_{i=1}^{n}\Tilde{H}_{i}(w)
\end{aligned}
\end{equation*}

%%% Local Variables:
%%% mode: latex
%%% TeX-master: "mainProject"
%%% End:

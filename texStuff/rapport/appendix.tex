\subsubsection{Convergence analysis} The restricted gap function
$\mathcal{G}_{D_{\vec v}, D_{\vec z}}$ is upper bounded by:
\begin{equation} \mathcal{G}_{D_{\vec w}, D_{\vec z}}(\overline{\vec w^{\tau}},
\overline{\vec z^{\tau}}) \leq \frac{\left( D_{\vec w} + D_{\vec z} \right)
L}{\tau + 1}
\label{eq:ub}
\end{equation}

In his proof on the convergence of the extragradient algorithm, Nesterov uses a
function $f_D$ instead of $\mathcal{G}_{D_{\vec w}, D_{\vec z}}$, where $f_D$ is
defined as:
\begin{equation}
f_D(\vec x) = \max_{\vec y \in \mathcal{Q}} \left \{ \langle g(\vec y), \vec x -
\vec y \rangle : d(\vec x, \vec y) \right \}
\end{equation}

where the set $\mathcal{Q}$ is the set of parameters and $g$ is a monotone
operator. We can already see a link between the function $f_D$ and the gap
$\mathcal{G}_{D_{\vec w}, D_{\vec z}}$. This comes out as:

\begin{equation}
  \begin{aligned}
    \mathcal{G}_{D_{\vec w},D_{\vec z}}(\vec w, \vec z) &= \sum_i \vec w^T \vec F_i \vec z_i^* - (\vec w^*)^T \vec F_i \vec z_i - \sum_i (\vec w^T - (\vec w^*)^T ) \vec f_i (\vec y_i) - \sum_i \vec c_i^T (\vec z_i - \vec z_i^*)\\
    &= \sum_i (\vec z_i^*)^T \vec F_i^T \vec w - (\vec w^*)^T \vec F_i \vec z_i - \sum_i \vec (\vec f_i (\vec y_i))^T (\vec w - \vec w^*) - \sum_i \vec c_i^T (\vec z_i - \vec z_i^*)\\
    &= \sum_i (\vec z_i^*)^T \vec F_i^T (\vec w - \vec w^*) - (\vec w^*)^T \vec F_i (\vec z_i - \vec z_i^*) - \sum_i (\vec f_i(\vec y_i))^T (\vec w - \vec w^*) - \sum_i \vec c_i^T (\vec z_i - \vec z_i^*)\\
    &=
    \begin{pmatrix}
      \begin{array}{c}
        \sum_i \vec F_i \vec z_i^*\\
	-\vec F_1^T \vec w^*\\
	\vdots\\
	-\vec F_m^T \vec w^*
      \end{array}
    \end{pmatrix}^T
    \begin{pmatrix}
      \begin{array}{c}
	\vec w - \vec w^*\\
	\vec z_1 - \vec z_1^*\\
	\vdots\\
	\vec z_m - \vec z_m^*
      \end{array}
    \end{pmatrix} -
    \begin{pmatrix}
      \begin{array}{c}
	\sum_i \vec f_i(\vec y_i)\\
	\vec c_1\\
	\vdots\\
	\vec c_m
      \end{array}
    \end{pmatrix}^T
    \begin{pmatrix}
      \begin{array}{c}
	\vec w - \vec w^*\\
	\vec z_1 - \vec z_1^*\\
	\vdots\\
	\vec z_m - \vec z_m^*
      \end{array}
    \end{pmatrix}
  \end{aligned}
\end{equation}

From this, we deduce that the function $g$ from the definition of $f_D$ corresponds to:
\begin{equation}
  g(\vec w, \vec z) =
  \begin{pmatrix}
    \begin{array}{c}
      \sum_i \vec F_i \vec z_i^*\\
      - \vec F_i^T \vec w^*\\
      \vdots\\
      - \vec F_m^T \vec w^*
    \end{array}
  \end{pmatrix} -
  \begin{pmatrix}
    \begin{array}{c}
      \sum_i \vec f_i (\vec y_i)\\
      \vec c_1\\
      \vdots\\
      \vec c_m
    \end{array}
  \end{pmatrix} = \vec F \vec u^* - \vec a
\end{equation}


It is constant and thus monotone as required by Nesterov's proof of the
convergence of the algorithm. Its Lipschitz constant $L$ is equal to $\max_{\vec
u \in \mathcal{U}} \lVert \vec F (\vec u - \vec u') \rVert_2 / \lVert \vec u -
\vec u' \rVert_2 \leq \lVert \vec F \rVert_2$. Of course, a point $\vec w, \vec
z$ that satisifies $\lVert \vec w \rVert_2 \leq D_{\vec w}$ and $\lVert \vec z
\rVert_2 \leq D_{\vec z}$ also satisifies $\lVert (\vec w, \vec z) \rVert_2 \leq
D$ when $D = \sqrt{D_{\vec w}^2 + D_{\vec z}^2}$ since $(\vec w, 0) \perp (0,
\vec z)$. It is then easy to see that $f_D \geq \mathcal{G}_{D_{\vec w}, D_{\vec
z}}$. Thus, the function $\mathcal{G}_{D_{\vec w}, D_{\vec z}}$ is upper bounded
by the right-hand side of equation \ref{eq:ub}. We can also observe that the
function $g(\vec w, \vec z)$ is exactly the gradient of the objective
$\mathcal{L}(\vec w, \vec z)$ at the point $\vec w, \vec z$.



\subsection{From classical Frank-Wolfe to more sophisticated variants}
\subsubsection{Classical Frank-Wolfe}
Consider a linear minimization oracle,
\begin{equation*}
\begin{aligned}
    &LMO_{\mathcal{A}}(\nabla f(x_{t}))\in \textit{argmin}_{s\prime\in\mathcal{A}}\langle s\prime, \nabla f(x_{t})\rangle
\end{aligned}
\end{equation*}
Starting with an active set consisting of only an initial feasible point $S^{0}=
\{x_{0}\}$, the Frank-Wolfe algorithm adds an "atom" $s_{t}=
LMO_{\mathcal{A}}(\nabla f(x_{t}))$, to the active set in a convex combination
with its elements while maintaining this combination sparse.
\subsubsection{Convergence results}
We define the duality gap
\begin{equation*}
\begin{aligned}
    &g(\alpha^{k})= \underset{s\in\mathcal{M}}{\textit{max}}\langle \alpha^{k}-s, \nabla f(\alpha^{k})\rangle
\end{aligned}
\end{equation*}
By first order convexity of the objective, we have
\begin{equation*}
\begin{aligned}
    &f(s)\geq f(\alpha^{k})+ \langle \alpha^{k}-s, \nabla f(\alpha^{k})\rangle\\
    &\Longrightarrow g(\alpha^{k})= -\underset{s\in\mathcal{M}}{\textit{min}}\langle \alpha^{k}-s, \nabla f(\alpha^{k})\rangle \geq f(\alpha^{k})- f^{*}
\end{aligned}
\end{equation*}
We can thus see that the duality gap gives us a computable optimality guarantee. \\

\textbf{Definition.} The curvature constant $C_{f}$ is given by the maximum relative deviation of the objective function f from its linear aaporximations, over the domain $\mathcal{M}$,

\begin{equation*}
\begin{aligned}
    &C_{f}= \underset{\underset{ \gamma\in[0,1], y=x+\gamma(s-x)}{x,s\in\mathcal{M}}}{sup}\frac{2}{\gamma^{2}}\Big(f(y)- f(x)- \langle y-x, \nabla f(x)\rangle\Big)
\end{aligned}
\end{equation*}
\\
Intuitively, the curvature constant can be seen as a measure of how flat the objective function is. For example, if the objective is linear, say $f(x)= ax+ b$ and $x\in[e,f]$ then $\nabla f(x)= a$ and the curvature constant is zero:
\begin{equation*}
\begin{aligned}
    &C_{f}= \frac{2}{\gamma^{2}}\Big(ay+ b- ax- b +(-ay +ax)\Big)= 0
\end{aligned}
\end{equation*}
Moreover $s=\textit{argmin}_{s\in[e,f]}\langle s, a\rangle= \frac{e}{a}$. Hence
we reach the minimum in one F-W step.\\ Thus, we can observe that for flatter
functions, that is with smaller curvature
constants, Frank-Wolfe should converge faster. \\

\textbf{Theorem.} The duality gap obtained in the $t^{th}$  iteraton of the Frank-Wolfe algorithm satisfies
\begin{equation*}
\begin{aligned}
    &g(x_{t})\leq 2\beta\frac{C_{f}}{t+2}(1+\delta)
\end{aligned}
\end{equation*}
Where $\beta= \frac{27}{8}$ and $\delta$ is the approximation error tolerated in the $LMO$. \\

\textbf{Definition.} A function $f$ has Lipschitz continuous gradient if:
\begin{equation*}
\begin{aligned}
    &\forall x,y \in dom(f), \exists L >0 \quad\textit{such that}\quad\\
    &||\nabla f(x) -\nabla f(y)||^2 \leq L||x - y||^2
\end{aligned}
\end{equation*}
\textbf{Theorem.} If a convex function $f$ on $C$ has Lipschitz gradient, i.e $||\nabla f(x)- \nabla f(y)||_{p}\leq L_{q}||x- y||_{p},\quad\forall x,y\in C$, then
\begin{equation*}
\begin{aligned}
      &C_{f}\leq L_{q}.\text{diam}_{p}^{2}(C)
\end{aligned}
\end{equation*} \\

\textbf{\textit{Proof.}} $f$ has Lipschitz gradient therefore by the fundamental descent lemma we have,
\begin{equation*}
\begin{aligned}
      &f(y)- f(x)- \langle y- x, \nabla f(x)\rangle \leq \frac{L_{q}}{2}||y- x||_{p}^{2}\\
      &C_{f} \leq \underset{\underset{x,s,y\in C}{y=(1-\gamma)x+\gamma s}}{max}\frac{2}{\gamma^{2}}\frac{L_{q}}{2}\underbrace{||y- x||_{p}^{2}}_{=\gamma^{2}||x- s||_{p}^{2}}\\
      &C_{f} \leq L_{q}\quad\underbrace{\underset{x,s\in C}{max}||x- s||_{p}^{2}}_{\overset{\Delta}{=}\textit{diam}_{p}^{2}(C)}\quad\quad\Box.
\end{aligned}
\end{equation*}
Therefore, assuming $\delta=0$, we get the following optimality certificate
\begin{equation*}
\begin{aligned}
      &f(\alpha^{k})- f^{*}\leq  g(\alpha^{k})\leq 2\beta\frac{C_{f}}{t+2}\leq 2\beta\frac{L_{q}.\text{diam}_{p}^{2}(C)}{t+2}
\end{aligned}
\end{equation*}
Thus, we see that the Frank-Wolfe algorithm has a sublinear convergence rate.
\subsubsection{Optimality in terms of sparsity of the iterates}
\textbf{Lemma.} For $f(x)= ||x||_{2}^{2}$ and $1\leq k\leq n$, it holds that
\begin{equation*}
\begin{aligned}
      &\textit{min}_{\underset{card(x)\leq
      k}{x\in\Delta_{n}}}f(x)= \frac{1}{k},\quad\textit{and}\\
      &g(x)\geq \frac{2}{k}\quad\forall x\in\Delta_{n}\quad\textit{s.t.}\quad card(x)\leq k.
\end{aligned}
\end{equation*}
By the first equality we have, for any vector $x$ s.t. $card(x)= k$, we get $g(x)\leq \frac{1}{k}- \frac{1}{n}$.
Thus, combining the upper and lower bound, we have that the sparsity (number of used atoms) by the Frank-Wolfe algorithm is worst case optimal.
% \begin{algorithm}[tb]
%    \caption{Frank-Wolfe}
%    \label{alg:example}
\begin{algorithmic}
   \STATE Let $\alpha\in\mathcal{M}$
   \FOR{$k=0$ {\bfseries to} $K$}
   \STATE {Compute $s={\textit{argmin}}_{s\in\mathcal{M}}\langle s, \nabla f(\alpha^{k})\rangle$}
   \STATE Let $\gamma = \frac{2}{k+2}$, or optimize $\gamma$ by line search
   \STATE Update $\alpha^{k+1}= (1-\gamma)\alpha^{k}+ \gamma s$
   \ENDFOR
\end{algorithmic}
%\end{algorithm}
\textbf{Theorem.} Given a convex, differentiable objective
$f:\mathcal{M}^{1}\times...\times\mathcal{M}^{n}\to\mathbb{R}$, where $\forall
i\in\{1..n\}$, each factor\quad $\mathcal{M}^{i}\subseteq\mathbb{R}^{n}$ is
convex and compact, if we are at a point $x$ such that $f(x)$ is minimized along
each coordinate axis, then $x$ is a global minimum.\\

As in coordinate descent, we minimize the objective function one coordinate
(block) at a time. At each iteration, BCFW picks the $i^{th}$ block (from $n$)
uniformly at random and updates the $i^{th}$ coordinate of the corresponding
weight, by calling the maximization oracle on the chosen block.
% \begin{algorithm}[tb]
%    \caption{Block-Coordinate Frank-Wolfe}
%    \label{alg:example}
\begin{algorithmic}
    \STATE {Let $w^{0}= w_{i}^{0}= \overline{w}^{0}= 0$, $l^{0}= l_{i}^{0}= 0$}\\
    \FOR {$k=0...K$}{
        \STATE{Pick $i$ at random in $\{1,...,n\}$}\\
        \STATE {Solve $y_{i}^{*}=\underset{y_{i}\in\mathcal{Y}_{i}}{\textit{max}}\quad H_{i}(y,w^{k})$}\\
        \STATE {Let $w_{s}= \frac{1}{n\lambda}\psi_{i}(y_{i}^{*})$, and $l_{s}= \frac{1}{n}L_{i}(y_{i}^{*})$}\\
        \STATE {Let $\gamma= \frac{\lambda(w_{i}^{k}-w_{s})^{T}w^{k}- l_{i}^{k}+ l_{s}}{\lambda||w_{i}^{k}-w_{s}||^{2}}$, and clip to $[0,1]$}\\
        \STATE {Update $w_{i}^{k+1}= (1-\gamma)w_{i}^{k}+ \gamma w_{s}$, and $l_{i}^{k+1}= (1-\gamma)l_{i}^{k}+ \gamma l_{s}$}\\
        \STATE {Update $w^{k+1}= w^{k}+ w_{i}^{k+1}- w_{i}^{k}$, and $l_{i}^{k+1}= (1-\gamma)l_{i}^{k}+ \gamma l_{s}$}\\
    }
    \ENDFOR
\end{algorithmic}
%\end{algorithm}
\subsubsection{Convergence Results}
\textbf{Definition.} Over each coordinate block $\mathcal{M}^{i}$, let the curvature be given by,
\begin{equation*}
\begin{aligned}
    &C^{(i)}_{f}=
\underset{\underset{\underset{\gamma\in[0,1]}{y=x+\gamma(s_{[i]}-x_{[i]})}}{x\in\mathcal{M},s_{i}\in\mathcal{M}^{i}}}{sup}\frac{2}{\gamma^{2}}\Big(f(y)-
f(x)- \langle y_{i}-x_{i}, \nabla_{i} f(x)\rangle\Big)
\end{aligned}
\end{equation*}
Where $x_{[i]}$ refers to the zero-padding of $i^{th}$ coordinate of $x$. And
let the global \emph{product curvature constant} be,
\begin{equation*}
\begin{aligned}
    &C^{\otimes}_{f}= \sum_{i=1}^{n}C^{(i)}_{f}
\end{aligned}
\end{equation*}
\textbf{Theorem.} For the dual structural SVM objective function over the domain
$\mathcal{M}= \Delta_{|\mathcal{Y}_{1}|}\times...
\times\Delta_{|\mathcal{Y}_{n}|}$, the total curvature constant
$C^{\otimes}_{f}$, on the product domain $\mathcal{M}$, is upper bounded by,
\begin{equation*}
\begin{aligned}
    &C^{\otimes}_{f} \leq \frac{4R^{2}}{\lambda n}
    \quad\textit{where} \quad R= \underset{i\in[n], y\in\mathcal{Y}_{i}}{max}||\psi_{i}(y)||_{2}
\end{aligned}
\end{equation*}
\textbf{\textit{Proof.}} By the second order convexity condition on $f$ at $y$, we have
\begin{align*}
    f(y)\leq f(x)+ \langle y_{i}-x_{i}, \nabla_{i} f(x)\rangle\\
    + (y- x)^{T}\nabla^{2}f(x)(y- x)\\
    f(y)- f(x)- \langle y_{i}-x_{i}, \nabla_{i} f(x)\rangle\\
    \leq (y- x)^{T}\nabla^{2}f(x)(y- x)
\end{align*}
\begin{equation*}
\begin{aligned}
    &C^{(i)}_{f}\leq\underset{\underset{\underset{\gamma\in[0,1]}{y=x+\gamma(s_{[i]}-x_{[i]})}}{x\in\mathcal{M},s_{i}\in\mathcal{M}^{i}}}{sup}\Big(f(y)- f(x)- \langle y_{i}-x_{i}, \nabla_{i} f(x)\rangle\Big)\\
    &\leq \underset{\underset{z\in[x,y]\subseteq\mathcal{M}}{x,y\in\mathcal{M},(y-x)\in\mathcal{M}^{[i]}}}{sup}(y- x)^{T}\nabla^{2}f(z)(y- x)\\
    &\textit{Moreover } \underset{\underset{z\in[x,y]\subseteq\mathcal{M}}{x,y\in\mathcal{M},(y-x)\in\mathcal{M}^{[i]}}}{sup}(y- x)^{T}\nabla^{2}f(z)(y- x)\\
    &= \lambda \underset{{x,y\in\mathcal{M},(y-x)\in\mathcal{M}^{[i]}}}{sup}(A(y- x))^{T}\nabla^{2}f(z)(A(y- x))
\end{aligned}
\end{equation*}
\begin{equation*}
\begin{aligned}
    &C^{(i)}_{f}\leq \lambda \underset{v,w\in A\mathcal{M}^{(i)}}{sup}||v- w||^{2}_{2}\leq \lambda \underset{v\in A\mathcal{M}^{(i)}}{sup}||2v||^{2}_{2}
\end{aligned}
\end{equation*}
Where $\forall v\in A\mathcal{M}^{(i)}$, $v$ is a convex combination of the
feature vectors corresponding to the possible labelings for the $i^{th}$ example
of the training data, such that $||v||_{2}\leq\textit{ the longest column of A}=
\frac{1}{n\lambda}R$. Therefore,
\begin{equation*}
\begin{aligned}
    &C^{\otimes}_{f}= \sum_{i=1}^{n}C^{(i)}_{f}\leq
4\lambda\sum_{i=1}^{n}\Big(\frac{1}{n\lambda}R\Big)^{2}=
\frac{4}{n\lambda}R^{2}\quad\Box
\end{aligned}
\end{equation*}
First, we observe that the curvature constant for BCFW is $n$ times smaller than
that of batch Frank Wolfe which is $\leq \frac{4}{\lambda}R^{2}$. Hence the $n$
times faster convergence rate of BCFW. \subsubsection{Tightening the bound}
\textbf{Definition.} Let $||.||$ be a norm on $\mathbb{R}^{n}$. The associated
dual norm, denoted $||.||_{*}$ is defined as,
\begin{equation*}
\begin{aligned}
    ||z||_{*}= \underset{z}{\textit{sup}} \{z^{T}x|\quad||x||\leq1\}
\end{aligned}
\end{equation*}
We denote the dual norm of $l_{p}$ by $l_{q}$. For $p=2$ we have $q=2$ and for $p=1$, $q=\infty$.
\textit{Problem.}\quad For $p= q= 2$ we get $\textit{diam}_{2}^{2}(C)= 2n$, and the Lipschitz constant $L_{q}$ is the largest eigenvalue of the hessian.
\begin{equation*}
\begin{aligned}
    &\lambda A^{T}A= \frac{1}{n^{2}\lambda}\Big(\langle \psi_{i}(y)- \psi_{j}(y\prime) \rangle\Big)_{(i,y),(j,y\prime)}\\
    &\textit{And say}\quad \langle \psi_{i}(y)- \psi_{j}(y\prime) \rangle \approx  1\quad\textit{for a lot of outputs, we get:}\\
    &\mathbbm{1}^{T}\mathbbm{1}\approx \mathbbm{1}\underbrace{diam_{2}(C)}_{=\sqrt{2n}}
\end{aligned}
\end{equation*}
Hence the largest eigenvalue the hessian, and therefore the Lipschitz constant,
can scale with the dimension of $A^{T}A$, i.e exponentially with the size of the
training data, rendering the bound above very loose, and thus of little
practical use. \\

\textit{Solution.}\quad Taking $p=1$ and therefore $q=\infty$, we get $L\textit{diam}^{2}(C)\approx \frac{4}{\lambda}R^{2}$.\\
\\
Combined with the the convergence results above, we get a sublinear convergence
rate for BCFW. And although subgradient methods converge at the same rate, BCFW
presents an adaptive stepsize and an indication as to when to terminate, making
it a more practical alternative.
\subsection{Randomized Away-step Frank-Wolfe}
A crucial assumption in constructing the BCFW is whether the domain is
block-separable. While this is true in the context of the structured SVM, this
leaves out important cases such as $l_{1}$ constrained optimization (e.g. lasso
type problems). \\
Moreover, while being an improvement on the classical variant by being
$n=\textit{size of the data}$ times cheaper per iteration, BCFW still converges
at a sublinear rate unlike the Away-step FW.\\

\textbf{The Randomized Away-steps Frank-Wolfe (RAWF)} finds a compromise between
the two variants. By subsampling a $\eta\in(0,1]$ portion of the domain
$\mathcal{A}$ in the $LMO$ and adding an away step at each iteration, we get a
linear convergence rate with cheaper oracle calls than that of the original F-W.
\subsubsection{Convergence results}
\textbf{Definition.}
Let the \textit{away curvature} $C^{A}_{f}$ and the \textit{geometric strong convexity} constants be, respectively
% \begin{equation*}
% \begin{aligned}
%     &C^{A}_{f}= \underset{\underset{\underset{\gamma\in[0,1]}{y=x+\gamma(s-
% x)}}{x,s,v\in\mathcal{M}}}{sup}\frac{2}{\gamma^{2}}\Big(f(y)- f(x)-
% \gamma\langle \nabla f(x), s- v\rangle\Big)\\ &\mu^{A}_{f}=
% \underset{x\in\matcal{M}}{\textit{inf}}\underset{\underset{\langle\nabla f(x),
% x^{*}-x\rangle <
% 0}{x^{*}\in\mathcal{M}}}{\textit{inf}}\frac{2}{\gamma^{A}(x,x^{*})^{2}}B_{f}(x,x^{*})\\
% &\texit{where}\quad\gamma^{A}(x,x^{*})=\frac{\langle -\nabla f(x), x^{*}-
% x\rangle}{\langle -\nabla f(x), s_{f}(x)- v_{f}(x)\rangle}\\
% \end{aligned}
% \end{equation*}
And $s_{f}, v_{f}(x)$ are the FW atom and away atom respectively, starting from $x$.\\

\textbf{Theorem.} Consider the set $\mathcal{M}=conv(\mathcal{A})$, with
$\mathcal{A}$ a finite set of extreme atoms, adter $T$ iterations of RAFW, we
have the following convergence rate
\begin{equation*}
\begin{aligned}
    &E\big[f(x_{T+1})\big]- f^{*}\leq \Big(f(x_{0})- f^{*}\Big).\Big(1- \eta^{2}\rho_{f}\Big)^{\textit{max}\{0,\lfloor\frac{T-s}{2}\rfloor\}}
\end{aligned}
\end{equation*}
With $\rho_{f}=
\frac{\mu^{A}_{f}}{4C^{A}_{f}},\quad\eta\frac{p}{|\mathcal{A}|}\quad$ and
$\quad s=|S_{0}|$.\\

\textbf{Proof sketch.} First we upper-bound $h_{t}=f(x_{t})- f^{*}$ by the
pairwise dual gap $\tilde{g}_{t}=\langle \tilde{s}_{t}- v_{t}\rangle$, then we
lower bound the progress $h_{t}- h_{t+1}$ by using the away curvature constant
in similar way to the proof in (Lacoste-Julien \& Jaggi, 2015, Theorem
8).$\Box$\\ \\
With the above theorem, we get
\begin{equation*}
\begin{aligned}
    &\underset{t\rightarrow \infty}{\textit{lim}} \frac{E f(x_{t+1})- f^{*}}{E f(x_{t})- f^{*}} \in \big(0,1\big)
\end{aligned}
\end{equation*}
Thus proving a linear convergence rate for the Randomized Away-steps Frank-Wolfe.
% \begin{algorithm}[tb]
%    \caption{Randomized Away-steps Frank-Wolfe}
%    \label{alg:example}
\begin{algorithmic}
    \STATE {Let $x_{0}=\sum_{v\in\mathcal{A}}\alpha^{(0)}_{v}$ with $s= |S_{0}|$, a subsampling parameter $1\leq p\leq |\mathcal{A}|$.}\\
    \FOR {$t=0...T$}{
        \STATE{Get $\mathcal{A}_{t}$ by sampling $min\{p,|\mathcal{A}\setminus S_{t}|\}$  elements uniformly from $|\mathcal{A}\setminus S_{t}|$}\\
        \STATE {Compute $s_{t}= LMO(\nabla f(x), S_{t}\bigcup \mathcal{A}_{t})$}\\
        \STATE {Let $d^{FW}_{t}= s_{t}- x_{t}$\quad\quad\textbf{RFW step}}\\
        \STATE {Compute $v_{t}= LMO(-\nabla f(x), S_{t})$}\\
        \STATE {Let $d^{A}_{t}= x_{t}- v_{t}$\quad\quad\quad\textbf{Away step}}\\
        \STATE {\textbf{if} $\langle-\nabla f(x_{t}), d^{FW}_{t}\rangle\geq \langle-\nabla f(x_{t}), d^{A}_{t}\rangle$\quad\textbf{then}}\\ \STATE{\quad\quad\quad$d_{t}=d^{FW}_{t}$ and $\gamma^{max}=1$}\\
        \STATE{\textbf{else}}\\
        \STATE{\quad\quad\quad $d_{t}=d^{A}_{t}$ and $\gamma^{max}=\frac{\alpha^{(t)}_{v_{t}}}{1-\alpha^{(t)}_{v_{t}}}$}\\
        \STATE{Let $x_{t+1}= x_{t}+ \gamma_{t}d_{t}$}\\
        \STATE{Let $S_{t+1}= \{v\in\mathcal{A}\quad s.t.\quad \alpha^{(t)}_{v^{}_{t}} > 0$\}}\\
    }
    \ENDFOR
\end{algorithmic}
%\end{algorithm}

%%% Local Variables:
%%% mode: latex
%%% TeX-master: "mainProject"
%%% End:

\subsection*{Proximal step operator} \label{proxstep}
We define the proximal step operator as follows:
\begin{equation}
  {T}_{\eta}(\vec u, \vec s) = \max_{\vec u \in {U}} \left \{
\langle \vec s, \vec u' - \vec u \rangle - \frac{1}{\eta} d(\vec u, \vec u')
\leq D \right \}
\end{equation}

The operator is useful to compute projections since when we have a strongly
convex function $h(\vec u)$, we can find its convex conjugate $h^*(\vec u) =
\max_{\vec u \in {U}} \left [ \langle \vec s, \vec u \rangle - h(\vec u)
\right ]$. From the definition of a strongly convex function, we have that:
\begin{equation}
  h(\vec u') \geq h(\vec u) + \langle \nabla h(\vec u), \vec u' - \vec u \rangle
+ \frac{\sigma}{2} \lVert \vec u' - \vec u \rVert^2
\end{equation}
where $\sigma$ is the strong convexity parameter. Rearranging, we can define an
upper bound on the squared norm of $\vec u' - \vec u$. This comes out as:
\begin{equation}
  d(\vec u', \vec u) \triangleq h(\vec u') - h(\vec u) - \langle \nabla h(\vec u), \vec u' - \vec u \rangle \geq \frac{\sigma}{2} \lVert \vec u' - \vec u \rVert^2
\end{equation}

The distance metric $d$ is called the Bregman divergence. The link between the
Bregman diverence and the proximal step operator is that if we are given the
function $h$ inside the definition of the proximal step update, this induces the
Bregman divergence, which in turn induces the update that is performed at each
iteration of the extragradient algorithm. For example, if we have $h(\vec u) =
\frac{1}{2} \lVert \vec u \rVert_2^2 $, the Bregman divergence becomes $d(\vec
u', \vec u) = \frac{1}{2} \lVert \vec u' - \vec u' \rVert_2^2$. We might wonder
why we care about the Bregman divergence when the definition still includes the
usual norm. After all, we still optimize the term $\langle \vec s, \vec u' -
\vec u \rangle - \frac{1}{\eta} d(\vec u', \vec u)$. This is because $h*$ is
differentiable at every point of its domain by the strong convexity of $h$.
Thus, it is easy to compute a projection in the usual fashion: we can compute
the derivative of the termn inside the projection operator and set it to 0. It
is impossible to do for matchings for example as the distance is not even
differentiable. We provide the steps to compute a projection:
\begin{equation}
  \vec s - \nabla_{\vec u'} d(\vec u', \vec u) = \vec s - \frac{1}{\eta}
\nabla_{\vec u'} d(\vec u, \vec u') = \vec s - \frac{1}{\eta} \left [\nabla
h(\vec u') - \nabla h(\vec u) \right]
\end{equation}

By setting this equation to 0, it is possible to recover the optimal $\vec y'$
when, let's say, $h(\vec u) = \frac{1}{2} \lVert \vec u' \rVert^2$.

%%% Local Variables:
%%% mode: latex
%%% TeX-master: "mainProject"
%%% End:

\subsection{classic optimal hyperplane problem}
Underdetermined!



\clearpage
\subsection{Large Margin Approach}
Taskar pioeneeried blabla max margin markov networks pgms are nice: \cite{taskarMaxMarginMarkovNetworks2004}
Tsochantaridis saw that combinatorial nature of certain problems virtually
require us to take the max margin / svm approach as more tradionational MLE
method require the partition function to be computed which is P\#-complete in
many interesting cases.\cite{tsochantaridisSupportVectorMachine}

For a dataset $ S = \{ (\vec x_i, \vec y_i) \}_{i=1}^{m} $,
where each $\vec x_i$ is an object with a structure (e.g. sequence of words in
french), we attempt to find the optimal parameter $\vec w$ of a linear classifier:

\begin{equation}
  \vec y_i = \argmax_{\vec y_i' \in \mathcal{Y}} \vec w^T \vec f(\vec x_i,\vec y_i')
  \label{eq1}
\end{equation}

The function $f$ is the feature mapping (which can be learned with modern deep
learning techniques) of a structured object with its
corresponding label $\vec y_i$. \textbf{Hinge Loss formulation}

\begin{equation}
  \min_{\vec w \in \mathcal{W}} \sum_i \max_{\vec y_i' \in \mathcal{Y}_i} \left[
\vec w^T \vec f_i(\vec y_i') + l_i(\vec y_i') \right] - \vec w^T \vec f_i(\vec
y_i)
\end{equation}

The parameters $\vec w$ are also regularized with parameter $\lambda$. Since we are
optimizing over $\vec y_i'$, we can drop the term from equation \ref{eq1} and we end
up with a loss-augmented inference problem inside the min function.




%%% Local Variables:
%%% mode: latex
%%% TeX-master: "mainProject"
%%% End:
